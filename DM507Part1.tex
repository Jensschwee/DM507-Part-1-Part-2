\documentclass{article}
\usepackage[utf8]{inputenc} %dansk tegnsæt
\usepackage[danish]{babel}  %Sætter dokumentet til dansk dvs at auto generat navne bliver danske
\usepackage{amsmath} %mat
\usepackage{amssymb}
\usepackage{amsfonts}
\usepackage{bold-extra}
\usepackage{textcomp}
\usepackage{graphicx}
\usepackage{amsmath}
\usepackage{fancyhdr}
\usepackage{listings}
\usepackage{color}

\definecolor{dkgreen}{rgb}{0,0.6,0}
\definecolor{gray}{rgb}{0.5,0.5,0.5}
\definecolor{mauve}{rgb}{0.58,0,0.82}

\lstset{frame=tb,
  language=Java,
  aboveskip=3mm,
  belowskip=3mm,
  showstringspaces=false,
  columns=flexible,
  basicstyle={\small\ttfamily},
  numbers=none,
  numberstyle=\tiny\color{gray},
  keywordstyle=\color{blue},
  commentstyle=\color{dkgreen},
  stringstyle=\color{mauve},
  breaklines=true,
  breakatwhitespace=true,
  tabsize=3
}
\pagestyle{fancy}
\fancyhead[CO,CE]{Projekt del 1 - DM507 - jesph13 og jenss12}
\begin{document}
{\centering
\huge
Projekt del 1\\
Algoritmer og datastrukturer (DM507)\\
\large
\bigskip
Institut for Matematik og Datalogi Syddansk Universitet, Odense\\
\bigskip
8. April 2015\\
\bigskip
Af Jesper Wohlert Hansen \& Jens Hjort Schwee\\
}
\newpage

\section*{Introduktion}
Denne rapport dækker over implementationer af datastukturerne \texttt{Priority Queue} og \texttt{Binary Tree} ved brug af hhv. en \texttt{Heap} og en \texttt{Dictionary} samt tilhørende algoritmer til behandling af disse strukturer. Disse implementationer er skrevet i Java og implementerer følgende interfaces:

\begin{verbatim}
public interface PQ {
    public Element extractMin();
    public void insert(Element e);
}
\end{verbatim}


\begin{verbatim}
public interface Dict {
    public void insert(int k);
    public int[] orderedTraversal();
    public boolean search(int k);
}
\end{verbatim}

De konkrete implementationer er baseret på pseudokode fra bogen \emph{Introduction to Algorithms, third edition}\footnote{http://mitpress.mit.edu/books/introduction-algorithms}.


Rapporten indeholder uddybende beskrivelser af de implementerede datastrukturer med tilhørende kodestumper. De mest trivielle kodestumper er undladt til fordel for korthed.

\newpage

\section*{Opgave 1}
Målet med denne opgave er at implementere en prioritetskø der implementerer interfacet \texttt{PQ}. Køen skal have en heap struktur, og skal indeholde og behandle objekter af klassen \texttt{Element}. Vi ønsker at fokusere på interface metoderne, samt private metodekald fra interface-metoderne.

\subsection*{extractMin()}

\begin{lstlisting}
// PQHeap.java
@Override
public Element extractMin() {
    Element min;
    min = heap[0];    
\end{lstlisting}
For at udtrække det mindste element fra heapen, benytter vi, at vi arbejder med en min-heap. Det vil sige, at vores mindste element altid vil være på plads 0 - \texttt{heap[0]}.
\begin{lstlisting}    
    heap[0] = heap[heapsize-1];
    heapsize--;
    minheapify(0);
    return min;
}
\end{lstlisting}
Herefter sætter vi det første element til at have samme værdi, som de sidste og reducerer heapsizen, således at vi kan heapifisere direkte fra første element. Dette fungerer da det største element nu er på roden af en min-heap, og hele heapen derfor skal reorganiseres.
\begin{lstlisting}
// PQHeap.java
private void minheapify(int i) {
    int smallest;
    int left = left(i);
    int right = right(i);
    Element swap;

    if (left <= heapsize && heap[left].key < heap[i].key)
        smallest = left;
    else
        smallest = i;
    if (right <= heapsize && heap[right].key < heap[smallest].key)
        smallest = right;
    if (smallest != i) {
        swap = heap[i];
        heap[i] = heap[smallest];
        heap[smallest] = swap;
        minheapify(smallest);
    }
}
\end{lstlisting}


Indsæt beskrivelse
\begin{lstlisting}
/**
 * Inserts the Element e into the heap
 *
 * param e - the element to be inserted
 */
@Override
public void insert(Element e) {
    heapsize++;
    heap[heapsize-1] = e;
    decreaseKey(heapsize-1);
}
\end{lstlisting}


\newpage


\section*{Opgave 2}
Formålet med denne delopgave er at implamentere træstruktur, denne skal udarbejdes på en sådan måde så den overholer \texttt{Dict} interfacet.\\
klassen \texttt{DictBinTree}. impalmenter interfacet, metoderne som der skal inføres ser ud som følgende:
\begin{lstlisting}
//DictBinTree search
public boolean search(int k) {
  Node x = root;
  while (x != null && k != x.key) {
        if (k < x.key) {
            x = x.left;
        } else {
            x = x.right;
        }
    }

    if (x == null) {
        return false;
    } else if (x.key == k) {
        return true;
    }
    return false;
}
\end{lstlisting}
\bigskip
Denne metode er lavet med udgangspunkt i bogens pseudokode.\\
Det som der foregår i denne er at først læses root, der søges igemmen træet indtil bundet er nået, hvis node'en man er ved er støre end k forsøttes der til venster ellers forsættes der til højre.\\
\newpage
\begin{lstlisting}
//DictBinTree insert
@Override
public void insert(int k) {
	int tempHeight;
	if (size != 0) {
		Node y = null;
       	Node x = root;
		tempHeight = 1;
		while (x != null) {
			y = x;
			if (k < x.key) {
				x = x.left;
			} else {
				x = x.right;
			}
			tempHeight++;
        }
		
		if (y == null) {
			root.key = k;
		} else if (k < y.key) {
			y.left = new Node(k);
		} else {
			y.right = new Node(k);
		}
            
		if (height < tempHeight) {
			height = tempHeight;
		}
	}
	else {
		root = new Node(k);
		height = 1;
	}
	size++;
}
\end{lstlisting}
\bigskip
Denne metode er lavet med udgangspunkt i bogens pseudokode, for indsæt i en træstruktur.\\
Det som der først teste der træet er tomt, hvis det er tomt oprettes den værdig som der skal indsættes som root, for træet. Ellers forsættes ned igemmen koden, hvor der findes den første leddigeplads hvor reglsættet for strukturen er overholdet, på denne plads indsættes k, i form af af ny Node.\\
Under indsætning af nye værdiger i træet vedligeholdes antallet af elementer heri og højden på træet. Dette kan ses i form af \texttt{size} og \texttt{height}.
\subsection*{orderedTraversal()}
Denne metode bruges til at udskrive alle elementerne i strukturen i numeisk rækkefølge.\\
Måden dette foregår er denne:
\begin{lstlisting}
//DictBinTree orderedTraversal
int[] outOrder;
public int[] orderedTraversal() {
    outOrder = new int[size];
    counter = 0;
    inOrderTreeWalk(root);
    return outOrder;
}
\end{lstlisting}
\bigskip
Denne metode bruges til at oprette et array med længden af elementer i træet. herefter kaldes \texttt{inOrderTreeWalk}, med root elementet for at fortælle hvor der skal startes med at søge.
\begin{lstlisting}
//DictBinTree inOrderTreeWalk
int counter = 0;
protected int[] inOrderTreeWalk(Node x) {
	if (x != null) {
		inOrderTreeWalk(x.left);
		outOrder[counter++] = x.key;
		inOrderTreeWalk(x.right);
	}
	return outOrder;
}
\end{lstlisting}
\bigskip
Denne metode er lavet med udgangspunkt i bogens pseudokode.\\
Metoden benytter sig af rekusive kald, først gås der til venster, herved køres den mindste side at træet først, og hør den højer kaldes skirver den selv til arrayet som bruges som output.\\
På denne måde sikkers at først skrives de mindste værdiger og herefter bliver de gradvis stører.
\subsection*{Node}
Måden hvorpå dataen gemmens i træet er via denne klasse \texttt{Node}. Den indeholder opnysninger om hvad key'en er på den og hvilke børn den har.
\begin{lstlisting}
//Node klasse
protected class Node {

        public Node left, right;
        public int key;
        
        protected Node(int key) {
            this.key = key;
            left = null;
            right = null;
        }
    }
\end{lstlisting}
\bigskip
\newpage
\section*{Opgave 3}
Koden for hvordan \texttt{Treesort} er implamenteret, den gør brug af datastrukturen \texttt{DictBinTree} som er et binært tree.\\
Ført læses indputter fra scanneren og gemmens i en arrayliste, herefter ligges disse tal i det binnære træ.\\
Til slut hentes alle talende ud igen af træ'et i sortede orden.
\begin{lstlisting}
//Treesort.java
List<Integer> numbers = new ArrayList<Integer>();
Scanner scan = new Scanner(System.in);
while (scan.hasNextInt()) {
	numbers.add(scan.nextInt());
}
Dict dict = new DictBinTree();
for (int number : numbers) {
	dict.insert(number);
}
for (int number : dict.orderedTraversal()) {
	System.out.println(number);
}
\end{lstlisting}
Koden for hvordan \texttt{Heapsort} er implamenteret, den gør brug af datastrukturen \texttt{PQHeap} som er en priotes kø.\\
Ført læses indputter fra scanneren og gemmens i en arrayliste, herefter ligges disse tal i det kø'en.\\
Til slut hentes alle talende ud igen af træ'et i sortede orden.
\begin{lstlisting}
//Heapsort.java
List<Integer> numbers = new ArrayList<Integer>();
Scanner scan = new Scanner(System.in);
while (scan.hasNextInt()) {
	numbers.add(scan.nextInt());
}
PQ pq = new PQHeap(numbers.size());
for (int number : numbers) {
	pq.insert(new Element(number, number));
}
for (int number: numbers) {
	System.out.println(pq.extractMin().key);
}
\end{lstlisting}
\newpage
\section*{Test}
Begge datastrukturer er testet på følgende måde:
\begin{itemize}
  \item console med indput
  \item console med redirection og output file  
  \item udleveret \texttt{TestProjectPartI} klasse
\end{itemize}
\bigskip
Datastrukeren vil virke på alle int's som ligger i int32 der er ikke testet for støre indput end dette. \texttt{Treesort} og \texttt{Heapsort} virker begge med disse indput.

\end{document}