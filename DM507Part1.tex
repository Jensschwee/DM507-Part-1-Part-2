\documentclass{article}
\usepackage[utf8]{inputenc} %dansk tegnsæt
\usepackage[danish]{babel}  %Sætter dokumentet til dansk dvs at auto generat navne bliver danske
\usepackage{amsmath} %mat
\usepackage{amssymb}
\usepackage{amsfonts}
\usepackage{bold-extra}
\usepackage{textcomp}
\usepackage{graphicx}
\usepackage{amsmath}
\usepackage{fancyhdr}
\usepackage{listings}
\usepackage{color}

\definecolor{dkgreen}{rgb}{0,0.6,0}
\definecolor{gray}{rgb}{0.5,0.5,0.5}
\definecolor{mauve}{rgb}{0.58,0,0.82}

\lstset{frame=tb,
  language=Java,
  aboveskip=3mm,
  belowskip=3mm,
  showstringspaces=false,
  columns=flexible,
  basicstyle={\small\ttfamily},
  numbers=none,
  numberstyle=\tiny\color{gray},
  keywordstyle=\color{blue},
  commentstyle=\color{dkgreen},
  stringstyle=\color{mauve},
  breaklines=true,
  breakatwhitespace=true,
  tabsize=3
}
\pagestyle{fancy}
\fancyhead[CO,CE]{Projekt del 1 - jesph13 og jenss12}
\begin{document}
{\centering 
\huge
Projekt del 1\\
Algoritmer og datastrukturer (DM507)\\
\large
\bigskip
Institut for Matematik og Datalogi Syddansk Universitet, Odense\\
\bigskip
8. April 2015\\
\bigskip
Af Jesper Wohlert Hansen \& Jens Hjort Schwee\\
}
\newpage

\section*{Introduktion}
Denne rapport dækker over implementationer af datastukturerne \texttt{Priority Queue} og \texttt{Binary Tree} ved brug af hhv. en \texttt{Heap} og en \texttt{Dictionary} samt tilhørende algoritmer til behandling af disse strukturer. Disse implementationer er skrevet i Java og implementerer følgende interfaces:

\begin{verbatim}
public interface PQ {
    public Element extractMin();
    public void insert(Element e);
}
\end{verbatim}


\begin{verbatim}
public interface Dict {
    public void insert(int k);
    public int[] orderedTraversal();
    public boolean search(int k);
}
\end{verbatim}

De konkrete implementationer er baseret på pseudokode fra bogen \emph{Introduction to Algorithms, third edition}\footnote{http://mitpress.mit.edu/books/introduction-algorithms}.


Rapporten indeholder uddybende beskrivelser af de implementerede datastrukturer med tilhørende kodestumper. De mest trivielle kodestumper er undladt til fordel for korthed.

\newpage

\section*{Opgave 1}

Indsæt beskrivelse
\begin{lstlisting}
	/**
     * Removes and returns the Element in the heap
     * with the lowest key.
     */
    @Override
    public Element extractMin() {
        Element min;
        min = heap[0];
        heap[0] = heap[heapsize-1];
        heapsize--;
        minheapify(0);
        return min;
    }    
\end{lstlisting}
Indsæt beskrivelse
\begin{lstlisting}
	/**
     * Inserts the Element e into the heap
     *
     * param e - the element to be inserted
     */
    @Override
    public void insert(Element e) {
        heapsize++;
        heap[heapsize-1] = e;
        decreaseKey(heapsize-1);
    }
\end{lstlisting}
\newpage
\section*{Opgave 2}
Indsæt beskrivelse
\begin{lstlisting}
	public boolean search(int k) {
        Node x = root;
        while (x != null && k != x.key) {
            if (k < x.key) {
                x = x.left;
            } else {
                x = x.right;
            }
        }

        if (x == null) {
            return false;
        } else if (x.key == k) {
            return true;
        }
        return false;
    }
\end{lstlisting}
Indsæt beskrivelse
\begin{lstlisting}
	public int[] orderedTraversal() {
        outOrder = new int[size];
        inOrderTreeWalk(root);
        return outOrder;
    }
\end{lstlisting}
\newpage
\section*{Opgave 3}
\end{document}